% Template LaTeX document for CSSR4Africa Deliverables
% Adapted from documents prepared by EPFL for the RobotCub project
% and subsequently by the University of Skövde for the DREAM project
%
% DV 28/06/2023

\documentclass{CSSRforAfrica}

\usepackage[hidelinks,colorlinks=false]{hyperref}
\usepackage[titletoc,title]{appendix}
\usepackage{latexsym}

\usepackage{dirtree}
\usepackage{booktabs}
\usepackage[table]{xcolor}
\usepackage{listings}

\usepackage{tcolorbox}
\usepackage{blindtext}  % For dummy text

\usepackage{tikz}
\usepackage{amsmath}
\usetikzlibrary{shapes.geometric, arrows, positioning, fit}

\newcommand{\blank}{~\\}
\newcommand{\checkbox}{{~~~~~~~\leavevmode \put(-7,-1.5){  \huge $\Box$  }}}

\begin{document}
\input{epsf}

%%
%% SHOULD NOT NEED TO BE CHANGED BEFORE THIS POINT
%% ------------------------------------------------
%%

\deliverable{D5.5.2.4}   
\title{D5.5.2.4 Integrated Text to Speech Conversion}   

\leadpartner{Carnegie Mellon University Africa}  
\partner{}                                % INSERT partner name: Carnegie Mellon University Africa or The University of the Witwatersrand

\revision{1.2}     
\deliverabledate{21/03/2025}    
\submissiondate{21/03/2025} 
\revisiondate{25/04/2025}      
\disseminationlevel{PU}
\responsible{Richard Muhirwa }      


%%
%% Create the titlepage
%%

\maketitle
 

\section*{Executive Summary}
%===============================================================
\label{executive_summary}
%%\addcontentsline{toc}{section}{Executive Summary}
 
Deliverable D5.5.2.4 presents the outcomes of Task 5.5.2.4, which integrates the outputs of Tasks 5.5.2.1 (English Text-to-Speech) and 5.5.2.3 (Kinyarwanda Text-to-Speech) into a unified system. This document outlines the results of each stage of the software development process, covering requirements definition, module specification, interface design, module design, testing, and implementation.
The Integrated Text-to-Speech system enables the Pepper robot to transform written text in either English or Kinyarwanda into spoken words through its internal speakers. This capability is fundamental to the robot's ability to verbally communicate with users in multiple languages, supporting a wide range of interactions from basic greetings to complex information delivery. The system accepts text input via ROS messages on the \texttt{/textToSpeech} topic and processes this text through the appropriate language-specific speech synthesis module based on configuration settings.
This report details the functional requirements, interface design specifications, module architecture, testing approach, and implementation instructions for the integrated Text-to-Speech conversion system. Testing results confirm that the Pepper robot's multilingual TTS system functions correctly, producing clear speech in both English and Kinyarwanda for a wide range of inputs, and maintaining stability even under stress conditions.

\newpage
 
 
%\graphicspath{{./figs/}}
\pagebreak
\tableofcontents
\newpage


\section{Introduction}
%===============================================================

The integrated TTS module developed in this deliverable combines the capabilities of both English and Kinyarwanda text-to-speech conversion into a single unified system. This integration allows for seamless switching between languages, enabling the Pepper robot to communicate effectively in multilingual environments.

The basic workflow of the integrated TTS system follows this pattern:
 Text input is received via a ROS service call to \texttt{/textToSpeech/say\_text}.

For English text, the system publishes to the \texttt{/speech}, and
for Kinyarwanda text, the Coqui TTS model generates a WAV file that is transferred to the robot via SSH and played using the ALAudioPlayer topic for processing by NAOqi.

 A success response is returned to the service caller

This dual-path architecture leverages the strengths of each approach: utilizing the robot's built-in capabilities for English speech while employing advanced neural TTS models for Kinyarwanda, where native support is not available in the NAOqi framework.

The implementation addresses practical challenges in multilingual robotics, such as integrating Python 3-based modern TTS frameworks with the Python 2-based NAOqi SDK, and handling the secure transfer and playback of generated audio files. This approach ensures optimal speech quality while maintaining a clean, unified interface for the rest of the robot's systems.
\\
This deliverable is structured as follows:

\textbf{Section 2} outlines the system's functional  and design considerations for the dual-language text-to-speech system. It details the service-based architecture and language-specific processing approaches. \textbf{Section 3} presents the module specification, detailing the text-to-speech conversion process for both English and Kinyarwanda languages. It explains how the system leverages NAOqi's built-in capabilities for English while employing the Coqui TTS neural model for Kinyarwanda speech synthesis. \textbf{Section 4} describes the interface design, covering the ROS service definition, configuration file format, and the bridge mechanism between Python 3 and Python 2 environments. It details how the system handles text input through the service interface and manages audio output through both direct topic publishing and SSH-based file transfer. \textbf{Section 5} details the module's architecture, including component interactions between the ROS node, NAOqi framework, and Coqui TTS library. \textbf{Section 6} contains a user manual with step-by-step instructions for configuration, and usage. It provides guidance on setting up the required dependencies, configuring language preferences, and invoking the text-to-speech service programmatically. \textbf{Section 7} provides an overview of the testing procedures, including unit tests for each language path and system validation under various network conditions and text input scenarios.
 
\newpage

%===============================================================
\section{Requirements Definition}

\subsection{Overview}
The objective of the integrated Text-to-Speech (TTS) system is to develop a software module that enables the robot to convert text input in either English or Kinyarwanda into natural-sounding speech. The system identifies the target language, processes the text accordingly, and generates appropriate audio output for playback on the robot's speakers. This TTS functionality serves as a crucial component for robot communication, enabling verbal interaction with users in multilingual contexts.

\subsection{Functional Specification}
The integrated TTS system operates as a ROS node called \texttt{textToSpeech} and provides the robot with the ability to vocalize text content in multiple languages. It supports two primary modes of operation:

\begin{enumerate}
    \item English Mode: Utilizes the NAOqi ALTextToSpeech system by publishing messages to the \texttt{/speech} topic, which is then processed by the robot's built-in speech capabilities.
    \item Kinyarwanda Mode: Employs the Coqui TTS model to generate speech waveforms that are then transferred to the robot via SSH and played using the ALAudioPlayer proxy.
\end{enumerate}

The system switches between these modes based on configuration settings while maintaining consistent quality standards across both languages.

\subsection{Text Processing and Speech Synthesis}
The system performs two key processes:

\begin{itemize}
    \item \textbf{Text Parsing and Preprocessing}: The system receives text input through a ROS service interface (\texttt{/textToSpeech/say\_text}), which accepts string messages. It processes the text according to the selected language mode.

    \item \textbf{Speech Synthesis}: For English, the system publishes the text to the \texttt{/speech} topic for native NAOqi processing. For Kinyarwanda, it utilizes the Coqui TTS model with conditioning audio to maintain consistent voice characteristics. The Kinyarwanda synthesis process creates a temporary WAV file, which is then securely copied to the robot via SSH and played using the ALAudioPlayer, with proper cleanup afterward.
\end{itemize}

The subsystem incorporates error handling mechanisms to provide meaningful feedback when errors occur, particularly when loading the Kinyarwanda speech synthesis components.

\newpage

\subsection{Configuration Management}
The robot's speech system is configured through a dedicated configuration file \\ (\texttt{textToSpeechConfiguration.ini}) that specifies:

\begin{enumerate}
    \item Language selection (English or Kinyarwanda)
    \item Verbose mode enabling or disabling for debugging purposes
    \item Robot IP address for connection
    \item Communication port for NAOqi services
\end{enumerate}

The system falls back to default configuration values if the configuration file cannot be read or processed, ensuring continuous operation even in the event of configuration errors.

\subsection{Inputs and Outputs}

\subsubsection{Service Provided}
The textToSpeech node provides a service for text vocalization:

\begin{table}[h]
    \centering
    \begin{tabular}{|p{4cm}|p{4cm}|p{6cm}|}
        \hline
      \rowcolor{lightgray}  \textbf{Service} & \textbf{Type} & \textbf{Description} \\
        \hline
        /textToSpeech/say\_text & text\_to\_speech/TTS & Service that accepts text messages and returns success status \\
        \hline
    \end{tabular}
    \caption{TTS Service Specification}
    \label{tab:tts_service}
\end{table}

\subsubsection{Topics Published (English Mode)}
When operating in English mode, the system publishes to:

\begin{table}[h]
    \centering
    \begin{tabular}{|p{3cm}|p{4cm}|p{7cm}|}
        \hline
      \rowcolor{lightgray}  \textbf{Topic} & \textbf{Type} & \textbf{Content} \\
        \hline
        /speech & std\_msgs/String & Text content to be spoken by NAOqi ALTextToSpeech \\
        \hline
    \end{tabular}
    \caption{TTS Published Topics}
    \label{tab:tts_topics}
\end{table}

\subsection{External Dependencies and Integration}
The TTS system relies on several external components for operation:

\begin{itemize}
    \item \textbf{NAOqi Framework}: Utilized for both direct English TTS via topic publishing and for audio playback in Kinyarwanda mode via the ALAudioPlayer proxy.
    \item \textbf{Coqui TTS}: Python library used for Kinyarwanda speech synthesis, requiring pre-trained models and configuration files located in the \texttt{model\_files} directory.
    \item \textbf{SSH Communication}: Secure file transfer to the robot for Kinyarwanda audio playback, requiring proper authentication and file management.
\end{itemize}

The system uses a Python 2 bridge script (\texttt{send\_and\_play\_audio.py}) to interface with the NAOqi framework for audio playback, bridging the gap between the modern Python 3 ROS node and the Python 2 NAOqi SDK requirements.

\subsection{Error Handling and Logging}
The system implements several mechanisms for error management:

\begin{itemize}
    \item \textbf{Configuration errors}: Fallback to default values with appropriate warnings
    \item \textbf{Model loading errors}: Critical error logging and process termination if TTS models cannot be loaded
    \item \textbf{Operation feedback}: Logging of text being spoken and the selected language
    \item \textbf{Service response}: Boolean success indication in the service response
\end{itemize}

All errors and operational messages are logged through the ROS logging infrastructure to facilitate debugging and system monitoring.


\newpage
\section{Function specification}
%===================================================================

\subsection{Functional Characteristics}

\subsubsection{Text-to-Speech Conversion}

The integrated TTS module consists of a unified ROS node (\texttt{textToSpeech}) that handles text-to-speech conversion for both English and Kinyarwanda languages. The language selection is determined by a configuration parameter in the \texttt{textToSpeechConfiguration.ini} file.

For English text, the system publishes messages to the \texttt{/speech} topic, which is then processed by the NAOqi framework to generate speech through the robot's internal speakers. For Kinyarwanda text, the system employs a TTS Synthesizer from the TTS library to generate a temporary WAV file, which is then transferred to the robot and played using the NAOqi ALAudioPlayer.

\subsubsection{Operation Modes}

The system supports two operation modes. In Normal Mode, standard operation occurs where text is converted to speech without additional output. In Verbose Mode, extended operation provides additional information logged for monitoring and debugging purposes.

\subsubsection{Audio File Generation and Playback}

For English TTS, the system publishes text to the \texttt{/speech} topic, which is then handled by the NAOqi driver for direct speech synthesis on the robot.

For Kinyarwanda TTS, the system follows a more complex process. First, it uses the TTS Synthesizer to generate audio data. Then, it creates a temporary WAV file. This file is transferred to the robot using SCP via a Python 2 script. Once transferred, the system plays the file using the NAOqi ALAudioPlayer, and finally removes the temporary file after playback \cite{casanova2022yourtts}.

\subsubsection{ROS Integration}

The module operates as a ROS node named \texttt{textToSpeech}. This node provides a service \\ \texttt{/textToSpeech/say\_text} to handle text-to-speech requests. It publishes to the \texttt{/speech} topic for English text and calls external scripts for Kinyarwanda text processing and playback.

\subsection{Inputs and Outputs}
The TTS module receives input text via a ROS service call, with the following specifications:

\begin{table}[h]
\centering
\begin{tabular}{|l|l|}
\hline
% \textbf{Parameter} & \textbf{Specification} \\
% \hline
\rowcolor{lightgray} \textbf{Parameter} & \textbf{Specification} \\ \hline
Format & UTF-8 encoded text string \\
\hline
Maximum Length & 1000 characters \\
\hline
Languages Supported & English, Kinyarwanda \\
\hline
Special Characters & Supported based on language requirements \\
\hline
Input Method & ROS service calls \\
\hline
\end{tabular}
\caption{Input Text Specifications}
\label{tab:input-specs}
\end{table}

\subsubsection{Inputs}

The system receives input through a ROS Service \texttt{/textToSpeech/say\_text}, which uses a custom \texttt{TTS} service type with a \texttt{message} field. This service accepts text strings to be converted to speech in either English or Kinyarwanda. Additionally, the system reads from a configuration file \texttt{textToSpeechConfiguration.ini}, which contains several parameters. These include the language selection (english or kinyarwanda), verboseMode setting (True or False), the robot's IP address, and port number.

\subsubsection{Outputs}

The system produces several types of output. For English text, it publishes to a ROS Topic \texttt{/speech} using the \texttt{std\_msgs/String} type. This contains the text to be spoken by the NAOqi framework. For Kinyarwanda, it generates an audio file - specifically a temporary WAV file that is transferred to the robot for playback. The system also provides logging through ROS logs for status information and debugging purposes.

\subsection{Dependencies}

\subsubsection{Common Dependencies}

The system relies on several common dependencies. These include ROS (with rospy), NAOqi drivers, Python 2 and Python 3.9 environments, and ConfigParser.

\subsubsection{English TTS Dependencies}

For English text-to-speech functionality, the system depends on the NAOqi framework's speech capabilities, which are accessed via the \texttt{/speech} topic.

\subsubsection{Kinyarwanda TTS Dependencies}

The Kinyarwanda functionality requires more extensive dependencies. The system needs the TTS library Synthesizer class and several pre-trained TTS model files. These include \texttt{model.pth} \\ (main model), \texttt{config.json} model configuration, \texttt{speakers.pth} speaker information, \\ \texttt{SE\_checkpoint.pth.tar} speaker encoder checkpoint, \texttt{config\_se.json} speaker encoder configuration, and \texttt{conditioning\_audio.wav} reference audio for voice characteristics \cite{kim2021conditional}. Additionally, it requires SSH/SCP access to the robot via sshpass and the NAOqi ALAudioPlayer.

%% This is for code formating {it specifies the background colors, the font to be used and the comments colors}

% Define colors
\definecolor{commentgreen}{RGB}{0, 155, 0}
\definecolor{backgroundcolor}{RGB}{240, 240, 249}
\definecolor{codeblack}{RGB}{0, 0, 0}
% \definecolor{stringblue}{RGB}{255, 165, 0}

% Configure listing style
\lstdefinestyle{commandstyle}{
  backgroundcolor=\color{backgroundcolor},
  basicstyle=\ttfamily\color{codeblack}\footnotesize,
  commentstyle=\color{commentgreen},
  % stringstyle=\color{stringblue},
  numbers=none,
  breaklines=true,
  breakatwhitespace=true,
  tabsize=2,
  showspaces=false,
  showstringspaces=false,
  frame=single,
  framesep=3pt,
  xleftmargin=5pt,
  xrightmargin=5pt,
  framexleftmargin=5pt,
  framexrightmargin=5pt,
  framextopmargin=3pt,
  framexbottommargin=3pt,
  framerule=0pt,
  % The important addition:
  morecomment=[l]{\#},  % This tells LaTeX that lines starting with # are comments
 morestring=[b]',      % Single-quoted strings
 morestring=[b]"       % Double-quoted strings
}


\newpage

\subsection{Execution Workflow}

The execution follows a defined workflow beginning with Node Initialization. During this phase, the system initializes the ROS node (\texttt{textToSpeech}), reads configuration from \\ \texttt{textToSpeechConfiguration.ini}, and sets up appropriate resources based on the selected language. For English, it creates a publisher to the \texttt{/speech} topic, while for Kinyarwanda, it initializes the TTS Synthesizer with model files. The system then sets up the ROS service \\ (\texttt{/textToSpeech/say\_text}).

Next in the workflow is Service Request Handling. The system receives text input through service calls and logs the request information.

The third phase involves Language-specific Processing. For English, the system publishes the text to the \texttt{/speech} topic. For Kinyarwanda, it generates speech using the TTS Synthesizer, saves it to a temporary WAV file, and calls the Python 2 script to transfer and play the audio.

The final phase is Audio Playback. For English, this is handled by the NAOqi framework. For Kinyarwanda, the system transfers the file to the robot, plays it using ALAudioPlayer, and then removes the temporary file.

\subsection{System Requirements}

The system has several requirements for proper operation. It requires a ROS workspace with NAOqi driver installed, a Python 3.9 environment for the main node, and a Python 2 environment for NAOqi compatibility. SSH access to the robot with password authentication is necessary, as are pre-trained TTS models for Kinyarwanda. The system also requires network connectivity to the robot.

\subsection{Limitations and Assumptions}

There are several limitations and assumptions in the current implementation. The system assumes the NAOqi driver is properly configured and running. For Kinyarwanda TTS, SSH access to the robot with the password "nao" is required. The robot must have sufficient disk space for temporary audio files. Audio quality may differ between the two languages due to different synthesis methods. Currently, the system supports only two languages: English and Kinyarwanda.

%% Interface Design Section
\newpage
%% Interface Design Section - Updated
\section{Interface Design}

\subsection{Directory Structure}

The integrated TTS module follows this directory structure:

\vspace{1cm}
\dirtree{%
.1 cssr4africa/.
.2 cssr\_system/.
.3 text\_to\_speech/.
.4 config/.
.5 textToSpeechConfiguration.ini.
.4 launch/.
.4 model\_files/.
.5 model.pth.
.5 config.json.
.5 speakers.pth.
.5 SE\_checkpoint.pth.tar.
.5 config\_se.json.
.5 conditioning\_audio.wav.
.5 conditioning\_audio2.wav.
.5 README.md.
.4 src/.
.5 send\_and\_play\_audio.py.
.5 textToSpeech.py.
.4 srv/.
.5 TTS.srv.
.4 README.md.
.3 CMakeLists.txt.
.3 package.xml.
.2 unit\_test/.
.3 ....
}
\vspace{1cm}

\subsection{ROS Service Definitions}

The system provides two primary ROS service interfaces:

\subsubsection{Text-to-Speech Service}

The main text-to-speech service allows for requesting speech synthesis with explicit language specification:

\newpage

\begin{lstlisting}[style=commandstyle]
# TTS.srv
string message  # Input text to synthesize
string language # Language model to utilize
---
bool success    # Operation status
\end{lstlisting}

This service is called as:
\begin{lstlisting}[style=commandstyle]
rosservice call /textToSpeech/say_text "message: 'Hello world' language: 'english'"
\end{lstlisting}


\subsection{ROS Topic Interface}

For English TTS, the system publishes to the following topic:

\begin{lstlisting}[style=commandstyle]
Topic: /speech
Type: std_msgs/String
\end{lstlisting}

The NAOqi driver listens to this topic and uses its built-in TTS capabilities to produce speech.

\subsection{Configuration Interface}

The system uses a simple INI-style configuration file with the following structure:

\begin{lstlisting}[style=commandstyle]
[DEFAULT]
verboseMode = False
ip = 172.29.111.230
port = 9559
\end{lstlisting}

The configuration is loaded at startup and affects the behavior of the system throughout its operation. Note that language selection is no longer handled through this configuration file but is instead managed through the ROS service interface and the culturalLanguageBase node.

\subsection{External Script Interface}

For Kinyarwanda TTS, a Python 2 script is called with the following arguments:

\begin{lstlisting}[style=commandstyle]
python2 send_and_play_audio.py <audio_file_path> <robot_ip> <robot_port>
\end{lstlisting}

This script transfers the audio file to the robot, plays it, and then removes the file.

\subsection{Error Handling}

The system provides error handling for the following scenarios:

\begin{enumerate}
    \item \textbf{Service Request Errors}: Errors during processing are logged, but the service continues to operate.
    \item \textbf{Language Selection Errors}: If an unsupported language is requested, an error is logged and the current language is maintained.
    \item \textbf{Synthesizer Initialization Errors}: If the TTS Synthesizer cannot be initialized, an error is logged and the system falls back to alternative methods where possible.
\end{enumerate}

%==================================================================
\newpage

\section{Module Design}

\subsection{Overall Architecture}

The integrated Text-to-Speech system provides a flexible architecture that supports both English and Kinyarwanda languages through a unified ROS service interface. The system consists of the following main components:

\begin{enumerate}
    \item\textbf{ROS Node}: \texttt{textToSpeech} Provides the service interface through \texttt{/textToSpeech/say\_text}
    \item \textbf{Language Handlers}:
    \begin{itemize}
        \item \textbf{English Handler}: Publishes to the \texttt{/speech} topic
        \item \textbf{Kinyarwanda Handler}: Uses TTS Synthesizer and external Python2 script for audio playback
    \end{itemize}
    \item \textbf{External Components}:
    \begin{itemize}
        \item \textbf{TTS Synthesizer}: Generates Kinyarwanda speech using pre-trained models
        \item \textbf{Audio Transfer \& Playback Script}: Python2 script that handles file transfer and playback on the robot
    \end{itemize}
\end{enumerate}

\subsection{Language Management}

During initialization, the TTS node loads all available language models to ensure they are ready for use. The actual language selection is managed by the BehaviorController, which invokes the language selection service to set the appropriate language based on information from the culturalLanguageBase node. This approach allows for dynamic language switching during operation without restarting the node.
The system interacts with the culturalLanguageBase node, which provides cultural context information including the preferred language for communication. The BehaviorController subscribes to updates from the culturalLanguageBase and invokes the language selection service as needed.

\subsection{Configuration Management}

The system loads configuration from an INI file during initialization:

\begin{lstlisting}[style=commandstyle]
def read_config():
    try:
        configParser = ConfigParser()
        config_file = os.path.join(root_dir, "text_to_speech/config/textToSpeechConfiguration.ini")
        configParser.read(config_file)
        config['language'] = configParser['DEFAULT']['language'].lower()
        config['verboseMode'] = configParser['DEFAULT']['verboseMode'] == "True"
        config['ip'] = configParser['DEFAULT']['ip']
        config['port'] = configParser['DEFAULT']['port']
    except:
        rospy.logwarn("Unable to read configuration file. Going with default values")
\end{lstlisting}

\newpage

The default configuration values are:
\begin{lstlisting}[style=commandstyle]
config = {
    'language':'english',
    'verboseMode':False,
    'ip':"172.29.111.240",
    'port':'9559'
}
\end{lstlisting}

\subsection{Module Initialization}

During initialization, the TTS node loads all necessary resources for both languages:

\begin{lstlisting}[style=commandstyle]
if __name__ == '__main__':
    read_config()
    try:
        from TTS.utils.synthesizer import Synthesizer
        synthesizer = Synthesizer(f"{model_files_dir}/model.pth",
                                f"{model_files_dir}/config.json",
                                tts_speakers_file=
                                f"{model_files_dir}/speakers.pth",
                                encoder_checkpoint=
                                f"{model_files_dir}/
                                SE_checkpoint.pth.tar",
                                encoder_config=
                                f"{model_files_dir}/config_se.json",
                                use_cuda=False
                                )
        pub = rospy.Publisher('/speech',String,queue_size=100)
    except:
        rospy.ERROR("Unable to load synthesizer for 'Kinyarwanda")
        exit(1)
    
    text_to_speech()
\end{lstlisting}

\subsection{English TTS Implementation}

For English TTS, the system uses a simple ROS publisher to send text to the \texttt{/speech} topic:

\begin{lstlisting}[style=commandstyle]
if language == 'english':
    rospy.loginfo(f"Saying '{message}' in {language}")
    pub.publish(message)
\end{lstlisting}

\newpage

\subsection{Kinyarwanda TTS Implementation}

For Kinyarwanda TTS, the system uses the TTS Synthesizer:

\begin{lstlisting}[style=commandstyle]
elif language == 'kinyarwanda':
    rospy.loginfo(f"Saying '{message}' in {language}")
    wav = synthesizer.tts(message, speaker_wav=f"{model_files_dir}/conditioning_audio.wav")
    
    with tempfile.NamedTemporaryFile(suffix=".wav", delete=False) as fp:
        synthesizer.save_wav(wav, fp)
    
    subprocess.run([python2_path, python2_script, fp.name, config['ip'], config['port']])
\end{lstlisting}

\subsection{Service Handler Implementation}

The TTS service handler processes incoming text-to-speech requests with language specification:

\begin{lstlisting}[style=commandstyle]
def say_text(request):
    message = request.message
    language = request.language.lower()
    if language not in supported_languages:
        rospy.logerr(f"Unsupported language: {language}")
        return TTSResponse(success=False)
    
    if language == 'english':
        rospy.loginfo(f"Saying '{message}' in {language}")
        pub.publish(message)
    elif language == 'kinyarwanda':
        rospy.loginfo(f"Saying '{message}' in {language}")
        wav = synthesizer.tts(message, speaker_wav=f"{model_files_dir}/conditioning_audio.wav")
        
        with tempfile.NamedTemporaryFile(suffix=".wav", delete=False) as fp:
            synthesizer.save_wav(wav, fp)
        
        subprocess.run([python2_path, python2_script, fp.name, config['ip'], config['port']])
    return TTSResponse(success=True)
\end{lstlisting}

\subsection{Paths and Configuration}

The system uses ROS package utilities to locate necessary files:

\begin{lstlisting}[style=commandstyle]
rospack = rospkg.RosPack()
root_dir = rospack.get_path('cssr_system')
model_files_dir = os.path.join(root_dir, "text_to_speech/model_files")
python2_path = '/usr/bin/python2'
python2_script = os.path.join(root_dir, "text_to_speech/src/send_and_play_audio.py")
supported_languages = ['english', 'kinyarwanda']
\end{lstlisting}


\section{User Manual}

\begin{lstlisting}[style=commandstyle]
# For Python 3.9
pip install rospkg TTS

# For Python 2
pip2 install pynaoqi

# System dependencies
sudo apt-get install sshpass

\end{lstlisting}

\begin{lstlisting}[style=commandstyle]
cd tts_combined
catkin_make

\end{lstlisting}

\begin{lstlisting}[style=commandstyle]
roslaunch naoqi_driver naoqi_driver.launch nao_ip:=172.29.111.230 network_interface:=wlp0s20f3

\end{lstlisting}


\begin{lstlisting}[style=commandstyle]
cd workspace/pepper_rob_ws
source devel/setup.bash
rosrun cssr_system textToSpeech.py

\end{lstlisting}

\begin{lstlisting}[style=commandstyle]
cd tts_combined
source devel/setup.bash
rosservice call /textToSpeech/say_text "message: 'muraho, murakaza neza, muri, robotics lab. '"

\end{lstlisting}


\begin{lstlisting}[style=commandstyle]
rosservice call /textToSpeech/say_text "message: 'Hello, welcome to robotics lab.'"
\end{lstlisting}

\begin{lstlisting}[style=commandstyle]
nano src/text_to_speech/config/textToSpeechConfiguration.ini
\end{lstlisting}

\begin{lstlisting}[style=commandstyle]
[DEFAULT]
language = english  # Change to "kinyarwanda" for Kinyarwanda
verboseMode = False
ip = 172.29.111.230
port = 9559
\end{lstlisting}

\begin{lstlisting}[style=commandstyle]
# Press Ctrl+C in Terminal 2 to stop the node
# Then restart it:
rosrun text_to_speech textToSpeech.py
\end{lstlisting}

\subsection{Common Issues}

\subsubsection{NAOqi Driver Connection Failure}

Check that the robot IP is correct
Ensure the network interface is correctly specified
Verify that the robot is powered on and connected to the network


\newpage
% =========================================================



\section{Unit Testing}

This report documents the testing of the integrated Text-to-Speech (TTS) system developed for the Pepper robot with support for both English and Kinyarwanda languages. The system was tested using the Robot Operating System (ROS) testing framework, with all tests executed successfully. 

The integrated TTS system provides language-specific speech synthesis capabilities for the Pepper robot. Key features include:
 Native English TTS using the robot's built-in capabilities.
 Custom Kinyarwanda TTS using language-specific processing.
 Dynamic language switching based on input parameters.
 ROS service interface for integration with other robot systems.

\subsection{Test Results: Audio Playback}
\begin{lstlisting}[style=commandstyle]
~/workspace/pepper_rob_ws$ rosrun unit_test integrated_tts_test_audio_playback.py

[Testcase: test_audio_file_exists] ... ok
[Testcase: test_script_execution] ... ok
[Testcase: test_script_existence] ... ok
---------------------------------------------------------------------
SUMMARY:
* RESULT: 
* TESTS: 3
* ERRORS: 0 []
* FAILURES: 0 []
\end{lstlisting}

\subsection{Test Results: TTS Service}
\begin{lstlisting}[style=commandstyle]
~/workspace/pepper_rob_ws$ rosrun unit_test integrated_tts_test_service.py

[Testcase: test_empty_message] ... ok
[Testcase: test_english_tts] ... ok
[Testcase: test_kinyarwanda_tts] ... ok
[Testcase: test_unsupported_language] ... ok
---------------------------------------------------------------------
SUMMARY:
* RESULT: 
* TESTS: 4
* ERRORS: 0 []
* FAILURES: 0 []
\end{lstlisting}

\subsection*{Test Results: TTS Integration Full Test}
\begin{lstlisting}[style=commandstyle]
~/workspace/pepper_rob_ws$ rosrun unit_test integrated_tts_full_test.py

=====================================================================
TTS INTEGRATION TEST
=====================================================================
This test will send various text messages to the TTS service in both English and Kinyarwanda.
The test will check if the robot correctly speaks all test messages.
Make sure the TTS service is running before proceeding.
Press Enter to begin testing...
Waiting for TTS service...
TTS service found!

=====================================================================
TEST 1: English TTS to Robot Integration
=====================================================================
Sending text: 'This is an English integration test. The robot should say this message.'
English message sent successfully
Waiting for 3.599999999999996 seconds for speech to complete...
Did you hear the English message correctly? (y/n): y
English first message test passed!
Sending text: 'This is a second English test message. Please confirm if you hear this clearly.'
English message sent successfully
Waiting for 4.2 seconds for speech to complete...
Did you hear the second English message clearly? (y/n): y
English second message test passed!

=====================================================================
TEST 2: Kinyarwanda TTS to Robot Integration
=====================================================================
Sending text: 'Muraho, ubu ni ubutumwa bwo kugenzura. Roboti igomba kuvuga ubu butumwa.'
{\color{terminalgreen}Kinyarwanda message sent successfully}
Waiting for 5.5 seconds for speech to complete...
Did you hear the Kinyarwanda message correctly? (y/n): y
Kinyarwanda message test passed!

=====================================================================
TEST 3: Language Switching Capability
=====================================================================
Sending English text: 'This is a test in English.'
Waiting for audio to play...
Sending Kinyarwanda text: 'Iri ni itsuzuma mu Kinyarwanda.'
Waiting for audio to play...
Sending English text: 'Now back to English.'
Waiting for audio to play...
Did all three messages play in the correct languages? (y/n): y
Language switching test passed!

=====================================================================
TEST SUMMARY
=====================================================================
Test 1: English TTS to Robot Integration - PASSED
Test 2: Kinyarwanda TTS to Robot Integration - PASSED
Test 3: Language Switching Capability - PASSED

All tests PASSED! The TTS is working correctly.

\end{lstlisting}


% ==================================================================
\newpage
\bibliographystyle{unsrt}
%================================================================
\bibliography{cognitive_systems.bib}                                     % REPLACE with correct filename
\addcontentsline{toc}{section}{References}



\pagebreak
\section*{Principal Contributors}
%===============================================================
\label{contributors}
\addcontentsline{toc}{section}{Principal Contributors}
The main authors of this deliverable are as follows (in alphabetical order).
\blank
~
\blank
David Vernon,  Carnegie Mellon University Africa.\\    % REPLACE with correct name and affiliation
Richard Muhirwa,  Carnegie Mellon University Africa.\\    % REPLACE with 
Tsegazeab Tefferi,  Carnegie Mellon University Africa                                                                              % REMOVE
% ==============================================================


\newpage
\section*{Document History}
%================================================================
\addcontentsline{toc}{section}{Document History}
\label{document_history}

\begin{description}

\item [Version 1.0]~\\
First draft. \\
Richard  Muhirwa. \\         
24 March 2025.                  

\item [Version 1.1]~\\
Updated the section 4 Interface Design subsection Directory structure. Added parent folder cssr4africa and cssr\_system and I include the internal README.md file.\\
Richard  Muhirwa. \\  
17 April 2025.

\item [Version 1.2]~\\
I updated the unit testing section by adding more tests including test audio playback and test TTS service. \\
Richard  Muhirwa. \\          
25 April 2025.

\end{description}

\end{document}

